% arara: lualatex: { shell: true, interaction: nonstopmode }
% arara: makeglossaries
% arara: biber
% arara: lualatex: { shell: true, interaction: nonstopmode }
% arara: lualatex: { synctex: true, shell: true, interaction: nonstopmode }

\providecommand{\toplevel}{..}
\providecommand{\sharedPath}{\toplevel/Shared}
\providecommand{\importPath}{\sharedPath/Imports}
\providecommand{\bibPath}{\sharedPath/bibFiles}
\providecommand{\assetPath}{\toplevel/Assets}
\providecommand{\luacodePath}{\toplevel/Shared/luaCode}

% Demo report and presentation prepared by Aaron English (humdrumcomet on github)
\documentclass{beamer}

\usetheme[progressbar=frametitle, numbering=fraction]{metropolis}

\input{\importPath/preamble}
\input{\importPath/glossary}


\title{Experimental \LaTeX~and Similar Toolchains}
\subtitle{Reproducible Documents and More}
\author{Aaron English}

\begin{document}
    \begin{frame}
        \titlepage
    \end{frame}
    \begin{frame}[noframenumbering, plain]
        \frametitle{Contents}
        \tableofcontents
    \end{frame}
    \section{Where Have we Been So Far?}
        \begin{frame}[t]
            \frametitle{\LaTeX~and the Extended Family}
            \begin{itemize}
                \item<1-> automation for elements\\
                    \only<2-7>{
                    \begin{itemize}
                        \item<2-7> TOC
                        \item<3-7> lists of Figures/Tables/Symbols/Constants
                        \item<4-7> numbering of Tables/Equations/Figures
                        \item<5-7> glossaries and acronyms
                        \item<6-7> symbols
                        \item<7> Bibliographies
                    \end{itemize}
                    }
                \item<8-> modularity\\
                    \only<9-12>{
                    \begin{itemize}
                        \item<9-12> reusing your work
                        \item<9-12> reduce duplication of effort
                        \item<10-12> preamble
                        \item<11-12> tables/equations/figures
                        \item<12> glossary/symbol entries
                    \end{itemize}
                    }
                \item<13-> All while maintaining high quality
                \item<13-> ...and minimizing effort
            \end{itemize}
            \only<14>{
            \centering
            \vfill
            \LARGE Can we push these ideas further?}
        \end{frame}
    \section{How Do We Work?}
        \begin{frame}
            \frametitle{How do we Use Computers?}
            \centering
            \includestandalone[mode=tex, width=0.8\textwidth]{\assetPath/Images/Tikz/workflow/workflow}
        \end{frame}
        \begin{frame}
            \frametitle{My Old Workflow - Designing a Circuit}
        \end{frame}
        \begin{frame}
            \frametitle{An Example - Designing a Circuit}
        \end{frame}
        \begin{frame}
          \frametitle{The Stratification of Toolchains}
          I don't want to fight with my tools, or work despite them, I want to weild them
        \end{frame}
        \begin{frame}
            \frametitle{An Alternative Approach}
        \end{frame}
        \begin{frame}
            \frametitle{Power User for All}
        \end{frame}
        \begin{frame}
            \frametitle{The Importance of Open Source}
        \end{frame}
        \begin{frame}
            \frametitle{The Importance of Open Source}
        \end{frame}
    \section{An Integrated Toolchain}
        \begin{frame}
            \frametitle{\hologo{LuaLaTeX}}
        \end{frame}
        \begin{frame}
            \frametitle{Emacs and Org-Mode}
        \end{frame}
    \section{Org-Mode Features}
        \begin{frame}
            \frametitle{A Preferrable Markdown Language}
        \end{frame}
        \begin{frame}
            \frametitle{Robust Conversion}
        \end{frame}
        \begin{frame}
            \frametitle{Unique Features}
        \end{frame}


    %     \begin{frame}[t,fragile]
    %         \frametitle<-5>{The \TeX~Procedure}
    %             \includestandalone[mode=tex, height=0.3\textheight]{\assetPath/Images/Tikz/texFlow/texFlow}
    %             \begin{onlyenv}<2>
    %                 \begin{table}
    %                     \centering
    %                     \caption{\LaTeX~Specific Editors}
    %                     \begin{tabular}{cccc}
    %                         & Linux & MacOS & Windows \\
    %                         TeXStudio & \checkmark & \checkmark & \checkmark \\
    %                         TeXMaker  & \checkmark & \checkmark & \checkmark \\
    %                         TeXnicCenter  & \checkmark & \checkmark & \checkmark \\
    %                     \end{tabular}
    %                 \end{table}
    %             \end{onlyenv}
    %             \begin{onlyenv}<3>
    %                 \begin{table}
    %                     \centering
    %                     \caption{Generic Text Editors with \LaTeX~Specific Extra's}
    %                     \begin{tabular}{cccc}
    %                         & Linux & MacOS & Windows \\
    %                         Emacs & \checkmark & \checkmark & \checkmark \\
    %                         Vim  & \checkmark & \checkmark & \checkmark \\
    %                         VSCode  & \checkmark & \checkmark & \checkmark \\
    %                         Sublime Text & \checkmark & \checkmark & \checkmark
    %                     \end{tabular}
    %                 \end{table}
    %             \end{onlyenv}
    %             \begin{onlyenv}<4>
    %                 \begin{table}
    %                     \centering
    %                     \caption{\TeX~Distributions for Different Operating Systems}
    %                     \centering
    %                     \begin{tabular}{cccc}
    %                         & Linux & MacOS & Windows \\
    %                         TeXLive & \checkmark & \checkmark & \checkmark \\
    %                         MacTeX  &  & \checkmark &  \\
    %                         MiKTeX  &  &  & \checkmark \\
    %                         ProTeXt &  &  & \checkmark
    %                     \end{tabular}
    %                 \end{table}
    %             \end{onlyenv}
    %             \frametitle<5>{Manually in Shell/Bash/Etc.}
    %             \begin{onlyenv}<5>
    %                 \begin{centering}
    %                     \begin{minted}{bash}
    %                         pdflatex --shell-escape --interaction=nonstopmode report
    %                         biber report
    %                         makeglossaries report
    %                         pdflatex --shell-escape --interaction=nonstopmode report
    %                         pdflatex --shell-escape --interaction=nonstopmode report
    %                     \end{minted}
    %                 \end{centering}
    %             \end{onlyenv}
    %             \frametitle<6>{Passing Commands at Compile Time}
    %             \begin{onlyenv}<6>
    %                 \begin{centering}
    %                     \begin{minted}{bash}
    %                         lualatex --shell-escape --interaction=nonstopmode "\\providecommand{\\iswhichmode}{draft} \\input{report}"
    %                         biber report
    %                         makeglossaries report
    %                         lualatex --shell-escape --interaction=nonstopmode "\\providecommand{\\iswhichmode}{draft} \\input{report}"
    %                         lualatex --shell-escape --interaction=nonstopmode "\\providecommand{\\iswhichmode}{final} \\input{report}"
    %                     \end{minted}
    %                 \end{centering}
    %             \end{onlyenv}
    %             \frametitle<7>{latexmk}
    %             \begin{onlyenv}<7>
    %                 \begin{centering}
    %                     \begin{minted}{bash}
    %                         latexmk -pdf report.tex
    %                     \end{minted}
    %                 \end{centering}
    %             \end{onlyenv}
    %             \frametitle<8>{ARARA}
    %             \begin{onlyenv}<8>
    %                 \begin{centering}
    %                     \begin{minted}{latex}
    %                         % arara: lualatex: { shell: true, interaction: nonstopmode }
    %                         % arara: makeglossaries
    %                         % arara: biber
    %                         % arara: lualatex: { shell: true, interaction: nonstopmode }
    %                         % arara: lualatex: { synctex: true, shell: true, interaction: nonstopmode }
    %                     \end{minted}
    %                     \begin{minted}{bash}
    %                         arara -v report.tex
    %                     \end{minted}
    %                 \end{centering}
    %             \end{onlyenv}
    %     \end{frame}
    % \section{External Tools}
    %     \begin{frame}[t,fragile]
    %         \frametitle{Matplotlib - Python Plotting}
    %         \begin{columns}[onlytextwidth]
    %             \begin{column}[T]{0.75\textwidth}
    %                 \begin{onlyenv}<1>
    %                     \inputminted{python}{\assetPath/Code/matplotlibexample.py}
    %                 \end{onlyenv}
    %                 \begin{onlyenv}<2>
    %                     \inputminted[firstline=29,lastline=45]{latex}{\assetPath/Code/examplePlot.tex}
    %                 \end{onlyenv}
    %                 \begin{onlyenv}<3>
    %                     \begin{minted}{latex}
    %                     \begin{figure}[H]
    %                     \begin{centering}
    %                     \includegraphics[width=0.5\textwidth] {\assetPath/Code/examplePlot.tex}
    %                     \caption{A figure produced with matplotlib}
    %                     \label{fig:test}
    %                     \end{centering}
    %                     \end{figure}
    %                     \end{minted}
    %                 \end{onlyenv}
    %                 \only<4>{\includestandalone[mode=tex, width=\textwidth]{\assetPath/Code/examplePlot}}
    %             \end{column}\hfill
    %             \begin{column}[T]{0.2\textwidth}
    %                 \centering
    %                 \includestandalone[mode=tex, width=\textwidth]{\assetPath/Images/Tikz/matplotlibFlow/matplotlibflow}
    %             \end{column}
    %         \end{columns}
    %     \end{frame}
    %     \begin{frame}[t,fragile]
    %         \frametitle{Sympy - Symbolic Math in Python}
    %         \begin{columns}[onlytextwidth]
    %             \begin{column}[T]{0.75\textwidth}
    %                 \begin{onlyenv}<1>
    %                     \inputminted{python}{\assetPath/Code/sympyexample.py}
    %                 \end{onlyenv}
    %                 \begin{onlyenv}<2>
    %                     \begin{minted}{latex}
    %                     \begin{equation}a^{b} e + \frac{c}{d}\end{equation}
    %                     \begin{equation}\frac{\partial}{\partial b} \left(a^{b} e + \frac{c}{d}\right) = a^{b} e \log{\left(a \right)}\end{equation}
    %                     \begin{equation}\int \left(a^{b} e + \frac{c}{d}\right)\, da = \frac{a c}{d} + \frac{a^{b + 1} e}{b + 1}\end{equation}
    %                     \end{minted}
    %                 \end{onlyenv}
    %                 \only<3>{
    %                     \begin{equation}a^{b} e + \frac{c}{d}\end{equation}
    %                     \begin{equation}\frac{\partial}{\partial b} \left(a^{b} e + \frac{c}{d}\right) = a^{b} e \log{\left(a \right)}\end{equation}
    %                     \begin{equation}\int \left(a^{b} e + \frac{c}{d}\right)\, da = \frac{a c}{d} + \frac{a^{b + 1} e}{b + 1}\end{equation}
    %                 }
    %             \end{column}\hfill
    %             \begin{column}[T]{0.2\textwidth}
    %                 \centering
    %                 \includestandalone[mode=tex, width=\textwidth]{\assetPath/Images/Tikz/sympyFlow/sympyflow}
    %             \end{column}
    %         \end{columns}
    %     \end{frame}
    %     \begin{frame}[t,fragile]
    %         \frametitle{Inkscape - Drawing and Manipulating Vector Graphics}
    %         \begin{columns}[onlytextwidth]
    %             \begin{column}[T]{0.75\textwidth}
    %                 \only<1>{
    %                     \begin{figure}
    %                     \centering
    %                     \includegraphics[height=0.7\textheight]{\assetPath/Images/InkscapeDemo/Acorn-RiscPC-PDF-Original}
    %                     \caption{Original vector graphic block diagram of the Acorn Risc PC \cite{ref:01}}
    %                     \label{fig:texsub}
    %                     \end{figure}
    %                 }
    %                 \begin{onlyenv}<2>
    %                     \includestandalone[mode=tex, width=0.9\textwidth]{\assetPath/Images/Tikz/inkscapeConversion/inkscapeConversion}
    %                 \end{onlyenv}
    %                 \begin{onlyenv}<3>
    %                     \inputminted[firstline=27,lastline=46]{latex}{\assetPath/Images/InkscapeDemo/Acorn-RiscPC-LaTeX.pdf_tex}
    %                 \end{onlyenv}
    %                 \begin{onlyenv}<4>
    %                     \begin{minted}{latex}
    %                     \begin{figure}[H]
    %                     \begin{centering}
    %                     \includestandalone[mode=tex, height=0.7\textheight]{\assetPath/Images/InkscapeDemo/Acorn-RiscPC-LaTeX-subbed-path}
    %                     \caption{Inkscape PDF+\TeX}
    %                     \label{fig:test}
    %                     \end{centering}
    %                     \end{figure}
    %                     \end{minted}
    %                 \end{onlyenv}
    %                 \only<5>{
    %                     \begin{figure}
    %                     \centering
    %                     \includestandalone[mode=tex, height=0.7\textheight]{\assetPath/Images/InkscapeDemo/Acorn-RiscPC-LaTeX-subbed-path}
    %                     \caption{Directly incorporating Inkscape PDF+\TeX~export of the Acorn Risc PC block diagram}
    %                     \label{fig:texsub}
    %                     \end{figure}
    %                 }
    %                 \only<6>{
    %                     \begin{figure}
    %                     \centering
    %                     \includestandalone[mode=tex, height=0.7\textheight]{\assetPath/Images/InkscapeDemo/Acorn-RiscPC-LaTeX-subbed-pdf-tex}
    %                     \caption{Inkscape PDF+\TeX~of the Acorn Risc PC block diagram with substitution of text via \hologo{LuaLaTeX}}
    %                     \label{fig:texsub}
    %                     \end{figure}
    %                 }
    %             \end{column}\hfill
    %             \begin{column}[T]{0.2\textwidth}
    %                 \centering
    %                 \includestandalone[mode=tex, width=\textwidth]{\assetPath/Images/Tikz/inkscapeFlow/inkscapeflow}
    %             \end{column}
    %         \end{columns}
    %     \end{frame}
    % \section{Backmatter}
    %     \begin{frame}[allowframebreaks]
    %         \frametitle{Bibliography}
    %         \printbibliography
    %     \end{frame}
    %     \begin{frame}[allowframebreaks]
    %         \frametitle{Acronyms}
    %         \printglossary[type=\acronymtype]
    %     \end{frame}
    %     % \begin{frame}[allowframebreaks]
    %     %     \frametitle{Glossary}
    %     %     \printglossary[type=main]
    %     % \end{frame}
    %     % \begin{frame}[allowframebreaks]
    %     %     \frametitle{Constants}
    %     %     \printglossary[type=constants, nonumberlist, nopostdot]
    %     % \end{frame}
    %     % \begin{frame}[allowframebreaks]
    %     %     \frametitle{Symbols}
    %     %     \printglossary[type=symbols]
    %     % \end{frame}
\end{document}
