% arara: lualatex: { shell: true, interaction: nonstopmode }
% arara: makeglossaries
% arara: biber
% arara: lualatex: { shell: true, interaction: nonstopmode }
% arara: lualatex: { synctex: true, shell: true, interaction: nonstopmode }

\providecommand{\toplevel}{..}
\providecommand{\sharedPath}{\toplevel/Shared}
\providecommand{\importPath}{\sharedPath/Imports}
\providecommand{\bibPath}{\sharedPath/bibFiles}
\providecommand{\assetPath}{\toplevel/Assets}
\providecommand{\luacodePath}{\toplevel/Shared/luaCode}

% Demo report and presentation prepared by Aaron English (humdrumcomet on github)
\documentclass{beamer}

\usetheme[progressbar=frametitle, numbering=fraction]{metropolis}

%%%%%%%%%%%%%%%%%%%%%%%%%%%%%%%%%%%%%%%%%%%%%%%%%%%%%%%%%%%%%%%%%%%%%%%%%%%%%%%%%%%%%%%%%%%%%%%%%%%%%%%%%%%%%%%%%%%%%%%%%%%%%%%%%%%%%%%%%%%%%%%%%%%%%%%%%%%%%%%%%
%%%%%%%%%% LOAD PACKAGES %%%%%%%%%%%%%%%%%%%%%%%%%%%%%%%%%%%%%%%%%%%%%%%%%%%%%%%%%%%%%%%%%%%%%%%%%%%%%%%%%%%%%%%%%%%%%%%%%%%%%%%%%%%%%%%%%%%%%%%%%%%%%%%%%%%%%%%%
%%%%%%%%%%%%%%%%%%%%%%%%%%%%%%%%%%%%%%%%%%%%%%%%%%%%%%%%%%%%%%%%%%%%%%%%%%%%%%%%%%%%%%%%%%%%%%%%%%%%%%%%%%%%%%%%%%%%%%%%%%%%%%%%%%%%%%%%%%%%%%%%%%%%%%%%%%%%%%%%%
\usepackage{geometry} % useful for defining page geometries
\usepackage{hyperref} % used for creating hyperlinks in documents. Both to the web and within the document itself
\usepackage[mode=build, subpreambles=false]{standalone} % for using the standalone package for import
\makeatletter
\@ifclassloaded{beamer}{
  \typeout{
    -------------------------------------------------
    -------------------------------------------------
    this file is a beamer file, skipping AMS packages
    -------------------------------------------------
    -------------------------------------------------
  }
}{
  \typeout{
    ----------------------------------------------------
    ----------------------------------------------------
    this file is not a beamer file, loading AMS packages
    ----------------------------------------------------
    ----------------------------------------------------
  }
  \usepackage[tbtags]{amsmath} % for typesetting math (American Mathematical Society)
  \usepackage{amsfonts} % fonts and mathematical symbols
  \usepackage{amssymb} % more mathematical symbols
}
\makeatother
% These packages aren't needed when compiling under LuaLaTeX
% \usepackage[utf8]{inputenc} % how to treat the written file (as utf8)
% \usepackage{morewrites} % important because with the glossaries, latex can use up its own hardcoded limit of files it can write out
% \usepackage[T1]{fontenc} % the encoding for the output file T1 is the most common, includes accents and many other commonly needed/used characters
%%%%%%%%%%%%%%%%%%%%%%%%%%%%%%%%%%%%%%%%%%%%%%%%%%%%%%%%%%%%%%%%%%%%%%%%%%%%%%%%%%%%%%%%%%%%%%%%%%%%%%%%%%%%%%%%%%%%%%%%%%%%%%%%%%%%%%%%%%%%%%%%%%%%%%%%%%%%%%%%%
\usepackage[style=ieee,backend=biber]{biblatex} % for handling bibliographies
\usepackage{float} % added control over
\usepackage{graphicx} % tools for inclusion of graphics
\usepackage{booktabs} % adding commands to improve the look of tables
\usepackage{csvsimple} % simplify table creation by importing .csv files directly
\usepackage{siunitx} % consistent notation and correct formatting of units
\usepackage{minted} % inclusion of code blocks with syntax highlighting and
\usepackage{xcolor} % to access the named colour LightGray
\usepackage{chemformula} % for writing chemical formulae
\usepackage[useregional]{datetime2} % facilitate date and time typesetting
\usepackage{catchfilebetweentags} % taking elements from between tags
\usepackage[debug, toc, section=section, acronym, symbols]{glossaries} % Glossaries package
\usepackage{tikz} % to produce tikz vector graphics images in your document
\usepackage{pgfplots} % to produce tikz/pgf vector graphic plots within your document
\usepackage[siunitx,american voltages, american currents, RPvoltages]{circuitikz} % circuits with tikz
\usepackage{hologo}
\usepackage{luacode}

%%%%%%%%%%%%%%%%%%%%%%%%%%%%%%%%%%%%%%%%%%%%%%%%%%%%%%%%%%%%%%%%%%%%%%%%%%%%%%%%%%%%%%%%%%%%%%%%%%%%%%%%%%%%%%%%%%%%%%%%%%%%%%%%%%%%%%%%%%%%%%%%%%%%%%%%%%%%%%%%%
%%%%%%%%%% PACKAGE SETUP %%%%%%%%%%%%%%%%%%%%%%%%%%%%%%%%%%%%%%%%%%%%%%%%%%%%%%%%%%%%%%%%%%%%%%%%%%%%%%%%%%%%%%%%%%%%%%%%%%%%%%%%%%%%%%%%%%%%%%%%%%%%%%%%%%%%%%%%
%%%%%%%%%%%%%%%%%%%%%%%%%%%%%%%%%%%%%%%%%%%%%%%%%%%%%%%%%%%%%%%%%%%%%%%%%%%%%%%%%%%%%%%%%%%%%%%%%%%%%%%%%%%%%%%%%%%%%%%%%%%%%%%%%%%%%%%%%%%%%%%%%%%%%%%%%%%%%%%%%
% minted
\definecolor{LightGray}{gray}{0.9}
\colorlet{PresentationBlack}{black!2}
\setminted{linenos=false, autogobble, fontsize=\tiny, breaklines=true, bgcolor=LightGray, breakbefore=\{} % global default options for minted
\setmintedinline{bgcolor={}, fontsize=\small} % global default options for inline minted

% tikz
\usetikzlibrary{math, arrows, circuits.ee.IEC, positioning, shapes.arrows, shapes.geometric, automata, fadings, overlay-beamer-styles}
\usepackage[draft]{tikzpeople}

% pgfplots
\pgfplotsset{compat=newest, compat/show suggested version=false}
\usepgfplotslibrary{groupplots}

% siunitx
\sisetup{
detect-family = true,
detect-weight = true,
per-mode = reciprocal,
input-digits = { 0123456789\pi\dots },
input-comparators = { <=>\approx\ge\geq\gg\le\leq\ll\sim\gtrsim\lesssim },
table-auto-round,
table-align-comparator = true,
}%
\DeclareSIUnit{\torr}{Torr} % Custom unit definition

% Bibliography .bib file location
\addbibresource[location=local]{\bibPath/references.bib}

% Add Constants list using glossary
\newglossary[cgls]{constants}{cstog}{cstig}{Constants}

% Alphabetize glossary and acronyms list
\makeglossaries

%%%%%%%%%%%%%%%%%%%%%%%%%%%%%%%%%%%%%%%%%%%%%%%%%%%%%%%%%%%%%%%%%%%%%%%%%%%%%%%%%%%%%%%%%%%%%%%%%%%%%%%%%%%%%%%%%%%%%%%%%%%%%%%%%%%%%%%%%%%%%%%%%%%%%%%%%%%%%%%%%
%%%%%%%%%% ADDITIONAL SETUP %%%%%%%%%%%%%%%%%%%%%%%%%%%%%%%%%%%%%%%%%%%%%%%%%%%%%%%%%%%%%%%%%%%%%%%%%%%%%%%%%%%%%%%%%%%%%%%%%%%%%%%%%%%%%%%%%%%%%%%%%%%%%%%%%%%%%
%%%%%%%%%%%%%%%%%%%%%%%%%%%%%%%%%%%%%%%%%%%%%%%%%%%%%%%%%%%%%%%%%%%%%%%%%%%%%%%%%%%%%%%%%%%%%%%%%%%%%%%%%%%%%%%%%%%%%%%%%%%%%%%%%%%%%%%%%%%%%%%%%%%%%%%%%%%%%%%%%
% A command for less cramped nested fractions
\newcommand\ddfrac[2]{\frac{\displaystyle #1}{\displaystyle #2}}

% Front matter, main matter, and back matter definitions (not needed for document class book or memoire)
\def\frontmatter{%
 \pagenumbering{roman}
 \setcounter{page}{1}
 \renewcommand{\thesection}{\roman{section}}
}%
\def\mainmatter{%
 \pagenumbering{arabic}
 \setcounter{page}{1}
 \setcounter{section}{0}
 \renewcommand{\thesection}{\arabic{section}}
}%
\def\backmatter{%
 \setcounter{section}{0}
 \renewcommand{\thesection}{\alph{section}}
}%

% taken from tikz  manual v3.1.10 sec. 106.5.3 Command for declaring new shapes pg 1149
\makeatletter
\pgfdeclareshape{document}{
\inheritsavedanchors[from=rectangle] % this is nearly a rectangle
\inheritanchorborder[from=rectangle]
\inheritanchor[from=rectangle]{center}
\inheritanchor[from=rectangle]{north}
\inheritanchor[from=rectangle]{south}
\inheritanchor[from=rectangle]{west}
\inheritanchor[from=rectangle]{east}
% ... and possibly more
\backgroundpath{% this is new
% store lower right in xa/ya and upper right in xb/yb
\southwest \pgf@xa=\pgf@x \pgf@ya=\pgf@y
\northeast \pgf@xb=\pgf@x \pgf@yb=\pgf@y
% compute corner of ‘‘flipped page’’
\pgf@xc=\pgf@xb \advance\pgf@xc by-10pt % this should be a parameter
\pgf@yc=\pgf@yb \advance\pgf@yc by-10pt
% construct main path
\pgfpathmoveto{\pgfpoint{\pgf@xa}{\pgf@ya}}
\pgfpathlineto{\pgfpoint{\pgf@xa}{\pgf@yb}}
\pgfpathlineto{\pgfpoint{\pgf@xc}{\pgf@yb}}
\pgfpathlineto{\pgfpoint{\pgf@xb}{\pgf@yc}}
\pgfpathlineto{\pgfpoint{\pgf@xb}{\pgf@ya}}
\pgfpathclose
% add little corner
\pgfpathmoveto{\pgfpoint{\pgf@xc}{\pgf@yb}}
\pgfpathlineto{\pgfpoint{\pgf@xc}{\pgf@yc}}
\pgfpathlineto{\pgfpoint{\pgf@xb}{\pgf@yc}}
\pgfpathlineto{\pgfpoint{\pgf@xc}{\pgf@yc}}
}
}
\makeatother

% Taken from TeX SE https://tex.stackexchange.com/questions/103688/folded-paper-shape-tikz
\tikzstyle{doc}=[%
draw,
thick,
align=center,
color=black,
shape=document,
minimum width=20mm,
minimum height=28.2mm,
shape=document,
inner sep=2ex,
% text width=10mm,
]

%Get rounded wire jumps
\tikzset{
    declare function={% in case of CVS which switches the arguments of atan2
        atan3(\a,\b)=ifthenelse(atan2(0,1)==90, atan2(\a,\b), atan2(\b,\a));},
        kinky cross radius/.initial=+.125cm,
        @kinky cross/.initial=+, kinky crosses/.is choice,
        kinky crosses/left/.style={@kinky cross=-},kinky crosses/right/.style={@kinky cross=+},
        kinky cross/.style args={(#1)--(#2)}{
        to path={
          let \p{@kc@}=($(\tikztotarget)-(\tikztostart)$),
              \n{@kc@}={atan3(\p{@kc@})+180} in
          -- ($(intersection of \tikztostart--{\tikztotarget} and #1--#2)!%
                 \pgfkeysvalueof{/tikz/kinky cross radius}!(\tikztostart)$)
          arc [ radius     =\pgfkeysvalueof{/tikz/kinky cross radius},
                start angle=\n{@kc@},
                delta angle=\pgfkeysvalueof{/tikz/@kinky cross}180 ]
          -- (\tikztotarget)}},
    onslide/.code args={<#1>#2}{%
        \only<#1>{\pgfkeysalso{#2}} % \pgfkeysalso doesn't change the path
    },
    myfading/.style n args={2}{
        postaction={
            decorate,
            decoration={
                markings,
                mark=between positions 0 and \pgfdecoratedpathlength-4pt step 0.2pt with {
                    \pgfmathsetmacro\myval{
                            multiply(
                                divide(
                                    \pgfkeysvalueof{/pgf/decoration/mark info/distance from start},
                                    \pgfdecoratedpathlength
                                ),
                                100
                            )
                    };
                    \pgfsetfillcolor{#2!\myval!#1};
                    \pgfpathcircle{\pgfpointorigin}{\pgflinewidth};
                    \pgfusepath{fill};
                },
                mark=at position 1 with {\arrow[#2, >=latex, auto]{>}},
            }
        }
    },
}
\usepackage[most]{tcolorbox}
\usepackage{tikzpagenodes}

\def\myblur{4}

\makeatletter
\newtcolorbox{blur}[1][]{%
  #1,
  enhanced,
  remember,
  frame hidden,
  interior hidden,
  fonttitle=\bfseries,
  coltitle=black,
  underlay={
    \begin{tcbclipframe}
      \begin{scope}[remember picture,overlay,inner sep=0pt]
        \fill[white] (current page.south west) rectangle (current page.north east);
        \foreach \x in {-10,-7.5,...,10}{
        \foreach \y in {-10,-7.5,...,10}{
          \node[opacity=0.01] at ([yshift=\y,xshift=\x]current page.center) {\includestandalone[mode=tex, width=\textwidth]{\assetPath/Images/Tikz/texfamilytree/texfamilytree}};
        }}
      \end{scope}
    \end{tcbclipframe}
   }
}
\makeatother

\def\checkmark{\tikz\fill[scale=0.4](0,.35) -- (.25,0) -- (1,.7) -- (.25,.15) -- cycle;}

% For using the standalone package and conditionally typesetting bibliography and glossaries
\newboolean{standaloneFlag}
\setboolean{standaloneFlag}{true}
% Create command to conditionally typeset a bibliography.
\newcommand{\standaloneBib}{%%
  \ifthenelse{\boolean{standaloneFlag}}{
    \clearpage
    \printbibliography[heading=bibintoc]
    \printglossary[type=symbols]
    \printglossary[type=constants]
    \printglossary[type=acronymtype]
    \printglossary[type=main]
  }{}}

% Code for Lua search and Replace

\directlua{
    path = "\luacodePath"
    demo = require(path.."/searchandreplace")
}
\newcommand\luastringsubs[3]{%
    \directlua{
        replacementsPreTable = [[\unexpanded{#1}]]
        filePathIn = [[#2]]
        filePathOut = [[#3]]
        subbedString = stringReplaceFromFile(replacementsPreTable, filePathIn, filePathOut)
    }%
}

%%%%%%%%%%%%%%%%%%%
%%%%%%%%%%%%%%%% Acronym Only
\newglossaryentry{vram}
{
    type=\acronymtype,
    name={VRAM},
    description={Video \glsentrydesc{ram}},
    first={\glsentrydesc{vram} (\glsentrytext{vram})}
}
\newglossaryentry{dram}
{
    type=\acronymtype,
    name={DRAM},
    description={Dynamic \glsentrydesc{ram}},
    first={\glsentrydesc{vram} (\glsentrytext{vram})}
}
\newglossaryentry{ram}
{
    type=\acronymtype,
    name={RAM},
    description={Random Access Memory},
    first={\glsentrydesc{ram} (\glsentrytext{ram})}
}
\newglossaryentry{asic}
{
  type=\acronymtype,
  name={ASIC},
  description={Application Specific \glsentrydesc{IC}},
  first={\glsentrydesc{asic} (\glsentrytext{asic})}
}
\newglossaryentry{ic}
{
  type=\acronymtype,
  name={IC},
  description={Integrated Circuit},
  first={\glsentrydesc{ic} (\glsentrytext{ic})}
}
\newglossaryentry{rom}
{
  type=\acronymtype,
  name={ROM},
  description={Read-Only-Memory},
  first={\glsentrydesc{rom} (\glsentrytext{rom})}
}
\newglossaryentry{risc}
{
  type=\acronymtype,
  name={RISC},
  description={Reduced Instruction Set},
  first={\glsentrydesc{risc} (\glsentrytext{risc})}
}
\newglossaryentry{ide}
{
  type=\acronymtype,
  name={IDE},
  description={Integrated Drive Electronics},
  first={\glsentrydesc{ide} (\glsentrytext{ide})}
}
\newglossaryentry{mux}
{
  type=\acronymtype,
  name={mux},
  description={multiplexer},
  first={\glsentrydesc{mux} (\glsentrytext{mux})}
}
\newglossaryentry{i/o}
{
  type=\acronymtype,
  name={I/O},
  description={Input/Output},
  first={\glsentrydesc{i/o} (\glsentrytext{i/o})}
}
\newglossaryentry{cpu}
{
  type=\acronymtype,
  name={CPU},
  description={Central Processing Unit},
  first={\glsentrydesc{cpu} (\glsentrytext{cpu})}
}
\newglossaryentry{iomd}
{
  type=\acronymtype,
  name={IOMD},
  description={\glsentrydesc{i/o} Memory Device},
  first={\glsentrydesc{iomd} (\glsentrytext{iomd})}
}
\newglossaryentry{rtc}
{
  type=\acronymtype,
  name={RTC},
  description={Real Time Clock},
  first={\glsentrydesc{rtc} (\glsentrytext{rtc})}
}
\newglossaryentry{i2c}
{
  type=\acronymtype,
  name={I\textsuperscript{2}C},
  description={Inter-Integrated Circuit},
  first={\glsentrydesc{i2c} (\glsentrytext{i2c})}
}
\newglossaryentry{i/f}
{
  type=\acronymtype,
  name={i/f},
  description={longform},
  first={\glsentrydesc{i/f} (\glsentrytext{i/f})}
}
\newglossaryentry{upc}
{
  type=\acronymtype,
  name={UPC},
  description={longform},
  first={\glsentrydesc{upc} (\glsentrytext{upc})}
}
\newglossaryentry{pc}
{
  type=\acronymtype,
  name={PC},
  description={Personal Computer},
  first={\glsentrydesc{pc} (\glsentrytext{pc})}
}


%%%%%%%%%%%%%%%%%%%
%%%%%%%%%%%%%%%% Glossary Only
\newglossaryentry{cpp}
{%
    name={C++},
    description={C++ is a programming language that can be used as an object oriented programming language, an imperative programming language, and still provide low-level memory control. Note: All C++ code used in this work is compiled under the C++11 standard}
}
%%%%%%%%%%%%%%%%%%%
%%%%%%%%%%%%%%%% Glossary and Acronym
% Self referencing glossary entry to minimize what needs to be edited between changes



\title{Experimental \LaTeX~and Similar Toolchains}
\subtitle{Reproducible Documents and More}
\author{Aaron English}

\begin{document}
    \begin{frame}
        \titlepage
    \end{frame}
    \begin{frame}[noframenumbering, plain]
        \frametitle{Contents}
        \tableofcontents
    \end{frame}
    \section{Where Have we Been So Far?}
        \begin{frame}[t]
            \frametitle{\LaTeX~and the Extended Family}
            \begin{itemize}
                \item<1-> automation for elements\\
                    \only<2-7>{
                    \begin{itemize}
                        \item<2-7> TOC
                        \item<3-7> lists of Figures/Tables/Symbols/Constants
                        \item<4-7> numbering of Tables/Equations/Figures
                        \item<5-7> glossaries and acronyms
                        \item<6-7> symbols
                        \item<7> Bibliographies
                    \end{itemize}
                    }
                \item<8-> modularity\\
                    \only<9-12>{
                    \begin{itemize}
                        \item<9-12> reusing your work
                        \item<9-12> reduce duplication of effort
                        \item<10-12> preamble
                        \item<11-12> tables/equations/figures
                        \item<12> glossary/symbol entries
                    \end{itemize}
                    }
                \item<13-> All while maintaining high quality
                \item<13-> ...and minimizing effort
            \end{itemize}
            \only<14>{
            \centering
            \vfill
            \LARGE Can we push these ideas further?}
        \end{frame}
    \section{How Do We Work?}
        \begin{frame}
            \frametitle{How do we Use Computers?}
            \centering
            \includestandalone[mode=tex, width=0.8\textwidth]{\assetPath/Images/Tikz/workflow/workflow}
        \end{frame}
        \begin{frame}
            \frametitle{My Old Workflow - Designing a Circuit}
        \end{frame}
        \begin{frame}
            \frametitle{An Example - Designing a Circuit}
        \end{frame}
        \begin{frame}
          \frametitle{The Stratification of Toolchains}
          I don't want to fight with my tools, or work despite them, I want to weild them
        \end{frame}
        \begin{frame}
            \frametitle{An Alternative Approach}
        \end{frame}
        \begin{frame}
            \frametitle{Power User for All}
        \end{frame}
        \begin{frame}
            \frametitle{The Importance of Open Source}
        \end{frame}
        \begin{frame}
            \frametitle{The Importance of Open Source}
        \end{frame}
    \section{An Integrated Toolchain}
        \begin{frame}
            \frametitle{\hologo{LuaLaTeX}}
        \end{frame}
        \begin{frame}
            \frametitle{Emacs and Org-Mode}
        \end{frame}
    \section{Org-Mode Features}
        \begin{frame}
            \frametitle{A Preferrable Markdown Language}
        \end{frame}
        \begin{frame}
            \frametitle{Robust Conversion}
        \end{frame}
        \begin{frame}
            \frametitle{Unique Features}
        \end{frame}


    %     \begin{frame}[t,fragile]
    %         \frametitle<-5>{The \TeX~Procedure}
    %             \includestandalone[mode=tex, height=0.3\textheight]{\assetPath/Images/Tikz/texFlow/texFlow}
    %             \begin{onlyenv}<2>
    %                 \begin{table}
    %                     \centering
    %                     \caption{\LaTeX~Specific Editors}
    %                     \begin{tabular}{cccc}
    %                         & Linux & MacOS & Windows \\
    %                         TeXStudio & \checkmark & \checkmark & \checkmark \\
    %                         TeXMaker  & \checkmark & \checkmark & \checkmark \\
    %                         TeXnicCenter  & \checkmark & \checkmark & \checkmark \\
    %                     \end{tabular}
    %                 \end{table}
    %             \end{onlyenv}
    %             \begin{onlyenv}<3>
    %                 \begin{table}
    %                     \centering
    %                     \caption{Generic Text Editors with \LaTeX~Specific Extra's}
    %                     \begin{tabular}{cccc}
    %                         & Linux & MacOS & Windows \\
    %                         Emacs & \checkmark & \checkmark & \checkmark \\
    %                         Vim  & \checkmark & \checkmark & \checkmark \\
    %                         VSCode  & \checkmark & \checkmark & \checkmark \\
    %                         Sublime Text & \checkmark & \checkmark & \checkmark
    %                     \end{tabular}
    %                 \end{table}
    %             \end{onlyenv}
    %             \begin{onlyenv}<4>
    %                 \begin{table}
    %                     \centering
    %                     \caption{\TeX~Distributions for Different Operating Systems}
    %                     \centering
    %                     \begin{tabular}{cccc}
    %                         & Linux & MacOS & Windows \\
    %                         TeXLive & \checkmark & \checkmark & \checkmark \\
    %                         MacTeX  &  & \checkmark &  \\
    %                         MiKTeX  &  &  & \checkmark \\
    %                         ProTeXt &  &  & \checkmark
    %                     \end{tabular}
    %                 \end{table}
    %             \end{onlyenv}
    %             \frametitle<5>{Manually in Shell/Bash/Etc.}
    %             \begin{onlyenv}<5>
    %                 \begin{centering}
    %                     \begin{minted}{bash}
    %                         pdflatex --shell-escape --interaction=nonstopmode report
    %                         biber report
    %                         makeglossaries report
    %                         pdflatex --shell-escape --interaction=nonstopmode report
    %                         pdflatex --shell-escape --interaction=nonstopmode report
    %                     \end{minted}
    %                 \end{centering}
    %             \end{onlyenv}
    %             \frametitle<6>{Passing Commands at Compile Time}
    %             \begin{onlyenv}<6>
    %                 \begin{centering}
    %                     \begin{minted}{bash}
    %                         lualatex --shell-escape --interaction=nonstopmode "\\providecommand{\\iswhichmode}{draft} \\input{report}"
    %                         biber report
    %                         makeglossaries report
    %                         lualatex --shell-escape --interaction=nonstopmode "\\providecommand{\\iswhichmode}{draft} \\input{report}"
    %                         lualatex --shell-escape --interaction=nonstopmode "\\providecommand{\\iswhichmode}{final} \\input{report}"
    %                     \end{minted}
    %                 \end{centering}
    %             \end{onlyenv}
    %             \frametitle<7>{latexmk}
    %             \begin{onlyenv}<7>
    %                 \begin{centering}
    %                     \begin{minted}{bash}
    %                         latexmk -pdf report.tex
    %                     \end{minted}
    %                 \end{centering}
    %             \end{onlyenv}
    %             \frametitle<8>{ARARA}
    %             \begin{onlyenv}<8>
    %                 \begin{centering}
    %                     \begin{minted}{latex}
    %                         % arara: lualatex: { shell: true, interaction: nonstopmode }
    %                         % arara: makeglossaries
    %                         % arara: biber
    %                         % arara: lualatex: { shell: true, interaction: nonstopmode }
    %                         % arara: lualatex: { synctex: true, shell: true, interaction: nonstopmode }
    %                     \end{minted}
    %                     \begin{minted}{bash}
    %                         arara -v report.tex
    %                     \end{minted}
    %                 \end{centering}
    %             \end{onlyenv}
    %     \end{frame}
    % \section{External Tools}
    %     \begin{frame}[t,fragile]
    %         \frametitle{Matplotlib - Python Plotting}
    %         \begin{columns}[onlytextwidth]
    %             \begin{column}[T]{0.75\textwidth}
    %                 \begin{onlyenv}<1>
    %                     \inputminted{python}{\assetPath/Code/matplotlibexample.py}
    %                 \end{onlyenv}
    %                 \begin{onlyenv}<2>
    %                     \inputminted[firstline=29,lastline=45]{latex}{\assetPath/Code/examplePlot.tex}
    %                 \end{onlyenv}
    %                 \begin{onlyenv}<3>
    %                     \begin{minted}{latex}
    %                     \begin{figure}[H]
    %                     \begin{centering}
    %                     \includegraphics[width=0.5\textwidth] {\assetPath/Code/examplePlot.tex}
    %                     \caption{A figure produced with matplotlib}
    %                     \label{fig:test}
    %                     \end{centering}
    %                     \end{figure}
    %                     \end{minted}
    %                 \end{onlyenv}
    %                 \only<4>{\includestandalone[mode=tex, width=\textwidth]{\assetPath/Code/examplePlot}}
    %             \end{column}\hfill
    %             \begin{column}[T]{0.2\textwidth}
    %                 \centering
    %                 \includestandalone[mode=tex, width=\textwidth]{\assetPath/Images/Tikz/matplotlibFlow/matplotlibflow}
    %             \end{column}
    %         \end{columns}
    %     \end{frame}
    %     \begin{frame}[t,fragile]
    %         \frametitle{Sympy - Symbolic Math in Python}
    %         \begin{columns}[onlytextwidth]
    %             \begin{column}[T]{0.75\textwidth}
    %                 \begin{onlyenv}<1>
    %                     \inputminted{python}{\assetPath/Code/sympyexample.py}
    %                 \end{onlyenv}
    %                 \begin{onlyenv}<2>
    %                     \begin{minted}{latex}
    %                     \begin{equation}a^{b} e + \frac{c}{d}\end{equation}
    %                     \begin{equation}\frac{\partial}{\partial b} \left(a^{b} e + \frac{c}{d}\right) = a^{b} e \log{\left(a \right)}\end{equation}
    %                     \begin{equation}\int \left(a^{b} e + \frac{c}{d}\right)\, da = \frac{a c}{d} + \frac{a^{b + 1} e}{b + 1}\end{equation}
    %                     \end{minted}
    %                 \end{onlyenv}
    %                 \only<3>{
    %                     \begin{equation}a^{b} e + \frac{c}{d}\end{equation}
    %                     \begin{equation}\frac{\partial}{\partial b} \left(a^{b} e + \frac{c}{d}\right) = a^{b} e \log{\left(a \right)}\end{equation}
    %                     \begin{equation}\int \left(a^{b} e + \frac{c}{d}\right)\, da = \frac{a c}{d} + \frac{a^{b + 1} e}{b + 1}\end{equation}
    %                 }
    %             \end{column}\hfill
    %             \begin{column}[T]{0.2\textwidth}
    %                 \centering
    %                 \includestandalone[mode=tex, width=\textwidth]{\assetPath/Images/Tikz/sympyFlow/sympyflow}
    %             \end{column}
    %         \end{columns}
    %     \end{frame}
    %     \begin{frame}[t,fragile]
    %         \frametitle{Inkscape - Drawing and Manipulating Vector Graphics}
    %         \begin{columns}[onlytextwidth]
    %             \begin{column}[T]{0.75\textwidth}
    %                 \only<1>{
    %                     \begin{figure}
    %                     \centering
    %                     \includegraphics[height=0.7\textheight]{\assetPath/Images/InkscapeDemo/Acorn-RiscPC-PDF-Original}
    %                     \caption{Original vector graphic block diagram of the Acorn Risc PC \cite{ref:01}}
    %                     \label{fig:texsub}
    %                     \end{figure}
    %                 }
    %                 \begin{onlyenv}<2>
    %                     \includestandalone[mode=tex, width=0.9\textwidth]{\assetPath/Images/Tikz/inkscapeConversion/inkscapeConversion}
    %                 \end{onlyenv}
    %                 \begin{onlyenv}<3>
    %                     \inputminted[firstline=27,lastline=46]{latex}{\assetPath/Images/InkscapeDemo/Acorn-RiscPC-LaTeX.pdf_tex}
    %                 \end{onlyenv}
    %                 \begin{onlyenv}<4>
    %                     \begin{minted}{latex}
    %                     \begin{figure}[H]
    %                     \begin{centering}
    %                     \includestandalone[mode=tex, height=0.7\textheight]{\assetPath/Images/InkscapeDemo/Acorn-RiscPC-LaTeX-subbed-path}
    %                     \caption{Inkscape PDF+\TeX}
    %                     \label{fig:test}
    %                     \end{centering}
    %                     \end{figure}
    %                     \end{minted}
    %                 \end{onlyenv}
    %                 \only<5>{
    %                     \begin{figure}
    %                     \centering
    %                     \includestandalone[mode=tex, height=0.7\textheight]{\assetPath/Images/InkscapeDemo/Acorn-RiscPC-LaTeX-subbed-path}
    %                     \caption{Directly incorporating Inkscape PDF+\TeX~export of the Acorn Risc PC block diagram}
    %                     \label{fig:texsub}
    %                     \end{figure}
    %                 }
    %                 \only<6>{
    %                     \begin{figure}
    %                     \centering
    %                     \includestandalone[mode=tex, height=0.7\textheight]{\assetPath/Images/InkscapeDemo/Acorn-RiscPC-LaTeX-subbed-pdf-tex}
    %                     \caption{Inkscape PDF+\TeX~of the Acorn Risc PC block diagram with substitution of text via \hologo{LuaLaTeX}}
    %                     \label{fig:texsub}
    %                     \end{figure}
    %                 }
    %             \end{column}\hfill
    %             \begin{column}[T]{0.2\textwidth}
    %                 \centering
    %                 \includestandalone[mode=tex, width=\textwidth]{\assetPath/Images/Tikz/inkscapeFlow/inkscapeflow}
    %             \end{column}
    %         \end{columns}
    %     \end{frame}
    % \section{Backmatter}
    %     \begin{frame}[allowframebreaks]
    %         \frametitle{Bibliography}
    %         \printbibliography
    %     \end{frame}
    %     \begin{frame}[allowframebreaks]
    %         \frametitle{Acronyms}
    %         \printglossary[type=\acronymtype]
    %     \end{frame}
    %     % \begin{frame}[allowframebreaks]
    %     %     \frametitle{Glossary}
    %     %     \printglossary[type=main]
    %     % \end{frame}
    %     % \begin{frame}[allowframebreaks]
    %     %     \frametitle{Constants}
    %     %     \printglossary[type=constants, nonumberlist, nopostdot]
    %     % \end{frame}
    %     % \begin{frame}[allowframebreaks]
    %     %     \frametitle{Symbols}
    %     %     \printglossary[type=symbols]
    %     % \end{frame}
\end{document}
